\documentclass[11pt,twoside,a4paper]{article}
\usepackage[margin=1in]{geometry}
\usepackage{verbatim}
\usepackage{graphicx}
\usepackage{url} % typeset URL's reasonably
\usepackage{listings}

\usepackage{pslatex} % Use Postscript fonts
\usepackage{algorithm}
\usepackage{algpseudocode}
%\usepackage[]{algorithm2e}
\usepackage{subcaption}
\usepackage{color}
\usepackage{multirow}
\usepackage{makecell}

\usepackage{mathptmx}
\usepackage{amsmath}

\usepackage{appendix}
%define some own functions
\newcommand{\tabincell}[2]{\begin{tabular}{@{}#1@{}}#2\end{tabular}} 
\def\D{\mathrm{d}}


\begin{document}

	\title{Training Spiking Restricted~Boltzmann~Machine by Minimising Contrastive~Divergence}
	\author{
	Qian~Liu
	\thanks{
	The author is with the School of Computer Science, University of Manchester, Manchester M13 9PL, U.K. 
	(e-mail:qian.liu-3@manchester.ac.uk}
	}
	\maketitle
	\thispagestyle{empty}

\begin{abstract}
	In order to implement training of Spiking Deep~Belief~Networks~(SDBNs) on SpiNNaker, this paper studies the layer-by-layer training of spiking Restricted~Boltzmann~Machine~(RBM) of a DBN.
	The study starts from understanding the original problem, Products of Experts~(PoE), which was solved by using Contrastive~Divergence~(CD).
	It involves utilising Markov~Chain~Monte~Carlo~(MCMC) sampling to present the distribution of a certain untraceable high-dimensional probability model function, e.g. PoE.
	Among these sampling algorithms, Gibbs method is introduced and used in PoE problem.
	Instead of minimising the original objective function of Kullback-Leibler divergence, the contrastive divergence is exploited to solve PoE.
	Then the study continues on applying CD to RBM.
	Finally, on-line learning methods only spiking neurons used are explored to train spiking RBM.
\end{abstract}

\section{Why CD(Contrastive Divergence)?\cite{hinton2002training,woodfordnotes}}
	The probability of a vector point $ \mathbf{x} $ is modelled by the function $f(\mathbf{x} \mid \Theta )$ given the model parameters $ \Theta $, and normalised by a partition function $Z( \Theta)$:
	
	\begin{equation}
	p(\mathbf{x} \mid \Theta ) = \dfrac{f(\mathbf{x} \mid \Theta )}{Z( \Theta)}
	\end{equation}
	where the partition function is defined as:
	\begin{equation}
	\label{equ:z_int}
	Z( \Theta) = \int f(\mathbf{x} \mid \Theta )\D\mathbf{x}, \text{when  $ \mathbf{x} $ is continuous, or}
	\end{equation}
	
	\begin{equation}
	\label{equ:z_dis}
	Z( \Theta) = \sum_{\mathbf{x}} f(\mathbf{x} \mid \Theta ), \text{when  $ \mathbf{x} $ is discrete.}
	\end{equation}
	Given a set of data points $ \mathbf{D}=(\mathbf{d}_1, \mathbf{d}_2, ..., \mathbf{d}_K) $ the purpose of learning is to tune the model parameter $ \Theta $ to fit the data $ \mathbf{D}  $. 
	The objective function is the probability product of all the independent data points of the data set, which is also called the likelihood function:
	\begin{equation}
	 L (\Theta \mid \mathbf{D}) = p(\mathbf{D} \mid \Theta ) = \prod_{k=1}^K p(\mathbf{d}_k \mid \Theta ) =  \prod_{k=1}^K\dfrac{f(\mathbf{d}_k \mid \Theta )}{Z( \Theta)}.
	\end{equation}
	 And the target is to maximise the likelihood given the data set $ \mathbf{D}  $, which equals to maximise the log of the probability product (the log-likelihood):
	\begin{equation}
	    \log  L (\Theta \mid \mathbf{D}) = \log p(\mathbf{D} \mid \Theta ) = \sum_{k=1}^K\log f(\mathbf{d}_k \mid \Theta ) - K \log Z( \Theta),
	\end{equation}
	or the average log-likelihood:
	\begin{equation}
	\label{equ:like}
		\hat{l} (\Theta \mid \mathbf{D}) =\frac{1}{K}\log  L (\Theta \mid \mathbf{D}) 
		=\frac{1}{K}\sum_{k=1}^K\log f(\mathbf{d}_k \mid \Theta ) - \log Z( \Theta).
	\end{equation}
	Imagine three different conditions (probability function) as following.
	
	\textbf{First}, $f(\mathbf{x} \mid \Theta )$ is the probability density function (pdf) of a normal distribution $\mathcal{N}(x \mid \mu, \sigma )$.
	Data vector $ \mathbf{x} $ is just a one dimensional data point, $x$.
	$ Z( \Theta) $ equals to 1, thus $p(x \mid \Theta ) = \mathcal{N}(x \mid \mu, \sigma )$.
	\begin{equation}
	\hat{l} (\Theta \mid D) =  
	\frac{1}{K} \sum_{k=1}^K \log \left[ \frac{1}{\sigma \sqrt{2\pi}} \exp(-\frac{(d_k-\mu)^{2}}{2\sigma^{2}}) \right]
	\label{pdf}
	\end{equation}
	To maximise Equation~\ref{pdf} is to find the Maximum Likelihood Estimation (MLE) of parameters $ \mu $ and $ \sigma $, by deriving from the partial differential equations when they are equal to 0. 
	\begin{equation}
	\left\{
	\begin{aligned}
	   &\dfrac{\partial \hat{l} (\Theta \mid D)}{\partial \mu}= \sum_{k=1}^K -\frac{1}{2\sigma^{2}}\dfrac{\partial (\mu-d_k)^{2}}{\partial \mu} = \sum_{k=1}^K -\frac{1}{\sigma^{2}}(\mu-d_k) = 0 \quad\\
	   &\dfrac{\partial \hat{l} (\Theta \mid D) }{\partial \sigma^{2}}= -\frac{K}{2\sigma^{2}}+\frac{1}{2\sigma^{4}}\sum_{k=1}^K d_k^{2} -\frac{\mu}{\sigma^{4}}\sum_{k=1}^K d_k + \frac{K\mu^{2}}{2\sigma^{4}} = 0 \quad\\
	\end{aligned}
	\right.
	\end{equation}
	\begin{equation}
	\left\{
	\begin{aligned}
	    &\mu= \frac{1}{K}\sum_{k=1}^K d_k  \quad\\
	    &\sigma^{2} = \frac{1}{K}\sum_{k=1}^K d_k^{2} - (\frac{1}{K}\sum_{k=1}^K d_k)^{2}
	\end{aligned}
	\right.
	\end{equation}
	$\hat{l} (\Theta \mid D)$ here is the function of two dimensional parameters $\mu$ and $\theta$, and searching the highest point in the parameter space ``is equivalent to being in the field on a clear, sunny day,''~\cite{woodfordnotes} seeing the point straight away.
	
	\textbf{Second}, the probability model function changes to be the sum of N normal distributions: 
	\begin{equation}
	f(x \mid \Theta ) = \sum_{i=1}^N\mathcal{N}(x \mid \mu_i, \sigma_i ).
	\end{equation}
	Derived from Equation~(\ref{equ:like}), the objective function is:
	\begin{equation}
	\hat{l} (\Theta \mid D) = \frac{1}{K}\sum_{k=1}^K \log \sum_{i=1}^N \mathcal{N}(d_k \mid \mu_i, \sigma_i ) - \log N,
	\end{equation}
	where $\log Z( \Theta)$ still equals a constant, but the partial differential equation of any parameter depends on other model parameters.
	It is very hard to solve equation set of log of sum, thus iteration methods are introduced, e.g., gradient descent method. %and expectation maximization (EM) algorithm
	Searching for a local maximum of the likelihood function in the parameter space starts with an initial point, either random or well selected.
	For each iteration, the partial derivatives for every dimension of the parameter point are calculated as the gradient.
	The gradient determines the decent direction of the space search, the next parameter point is one step $ \eta $  towards the direction or is the highest point found by line search.
	
	The gradient descent method is equivalent to ``being in the field at night with a torch.''~\cite{woodfordnotes}.
	And then the descent direction is estimated and chosen by using the torch to see the relative heights of the field a short distance in each direction.
	Because partial differential equation of any parameter depends on other model parameters, we can only see the gradient for a small area.
	The search will follow the direction by walking one step or a certain distance (e.g. line search lowest point), and then start a new iteration.
	

\subsection{PoE Problem}
	\textbf{Finally}, the probability model function becomes the product of N normal distributions: 
	\begin{equation}
	f(x \mid \Theta ) = \prod_{i=1}^N\mathcal{N}(x \mid \mu_i, \sigma_i ),
	\end{equation}
	where the partition (normalisation) function $Z( \Theta)$ is no longer a constant, but varies accordance to all the parameters.
	Essentially, the integration of the probability model, see Equation~(\ref{equ:z_int}) and~(\ref{equ:z_dis}), is usually algebraically intractable.
	We have to use numerical integration method to evaluate the Equation~(\ref{equ:like}), whose partial derivative is (we are using vectors to generalise the problem):
	\begin{equation}
	\label{equ:part}
	\begin{aligned}
	\dfrac{\partial \hat{l} (\Theta \mid \mathbf{D})}{\partial \theta} 
	& = \frac{1}{K} \dfrac{\partial \sum_{k=1}^K\log f(\mathbf{d}_k \mid \Theta )}{\partial \theta} - \dfrac{\partial \log Z( \Theta)}{\partial \theta}\\
	& =  \frac{1}{K}\sum_{k=1}^K \dfrac{\partial \log f(\mathbf{d}_k \mid \Theta)}{\partial \theta} - \int p(\mathbf{x} \mid \Theta) \dfrac{\partial \log f(\mathbf{x} \mid \Theta)}{\partial \theta} \D \mathbf{x}\\
	& = \left \langle \dfrac{\partial \log f(\mathbf{d} \mid \Theta)}{\partial \theta}\right \rangle_{\mathbf{D}} -\left \langle \dfrac{\partial \log f(\mathbf{c} \mid \Theta)}{\partial \theta}\right \rangle_{\mathbf{C} \sim p(\mathbf{x} \mid \Theta)}  \\
	&=\left \langle \dfrac{\partial \log f(\mathbf{x} \mid \Theta)}{\partial \theta}\right \rangle_{\mathbf{X}_{data}} - \left \langle \dfrac{\partial \log f(\mathbf{x} \mid \Theta)}{\partial \theta}\right \rangle_{\mathbf{X}_{model}},
	\end{aligned}
	\end{equation}
	where  $ <\cdot>_x $ denotes the mean expectation of $ \cdot $ given distribution of $x$.
	The first term of the right-hand side is easy to get with the given data set $ \mathbf{D} $, and the second term can be approximated by generating data samples $ \mathbf{C} $ according to $ p(\mathbf{x} \mid \Theta) $.
	These generative samples is called ``fantasy data'' and can be generated using Monte Carlo Markov Chain (MCMC) sampling, see section~\ref{sec:mcmc}.
	The detailed derivation process can be found in Appendix~\ref{app:part}.
	Although in this example the integration of product of normal distribution is still tractable, it is also helpful to use numerical integration.

	Go back to the metaphor of the parameter field, solving PoE problem is like searching the lowest point in a completely dark night without a torch.
	The computation of Equation~(\ref{equ:part}) is to ``feel the gradient of the field under our feet''.~\cite{woodfordnotes}.
	%The motivation underlining Contrastive Divergence algorithm is to boost the training speed of a Markov Chain in order to represent the distribution of a PoE (Product of Expert) model.
	%Thus the sampling can be followed using this trained Markov Chain model. 
\subsection{MCMC Sampling}
	\label{sec:mcmc}
	In order to solve the integration of algebraically intractable equations we can use numerical integration to approximate.
	One of the popular method is Monte Carlo integration:
	\begin{equation}
	\int_{a}^{b} f(x) \D x = \int_{a}^{b}\frac{f(x)}{q(x)}q(x)\D x = \dfrac{1}{N}\sum_{i=1}^{N}\frac{f(x_i)}{q(x_i)}.
	\end{equation}
	The integration of $ f(x) $ transforms to the integration of a new function $ F(x) = f(x)/q(x)  $ times its probability function $ q(x) $.
	It could be approximated by sampling N data points $ x_i $ according to the probability distribution $ q(x) $, and calculate the average of $ F(x_i) $ as $ <F(x)>_{q(x)}$.
	So the main question following is how to sample from a probability distribution.
	
	MCMC algorithm was proposed by Metropolis in 1953 and it became a wide-used sampling method.
	The stationary distribution $ \pi $ exists when every two nodes in a Markov Chain are connected regardless of the initial state distribution $ \pi_0 $:
	\begin{equation}
	\begin{aligned}
		&\pi(j) = \sum_{i=1}^{\infty}\pi(i)P_{ij} \\
		&\pi P = \pi,
	\end{aligned}
	\end{equation}
	where $ P $ is the transition probability matrix, and $ \pi $ is in the state space of a MC and the sum, $ \sum_{i=1}^{\infty}\pi(i) $,of a state distribution is 1.
	Thus based on the useful theorem of MC, sampled sequence $ \{x_0, x_1, ..., x_n, ... \}$ from a MC complies with its stationary distribution $ \pi(x) $.
	Metropolis stated that if a MC has a stationary distribution, $ q(x) $, which is exactly needed to sample from, then we can easily obtain a sample sequence along the MC according to the transformation probability matrix $ P $.
	Here so far we are describing the MC with discrete states, however the it also applies to continuous $ \pi $ and $ P $.
	The problem here is to build $ P $ to make the stationary distribution equal to the required probability, $ \pi(x) = q(x) $.
	
	So the other useful theorem (detailed balance) lies here, if an aperiodic MC is reversible: $\pi (i) P_{ij} = \pi (j) P_{ji},$ then $ \pi $ is the stationary distribution.
	It is a stronger condition than the previous theorem, so most of the MCs are not generally eligible:
	\begin{equation}
	\pi (i) P_{ij} \neq \pi (j) P_{ji}.
	\end{equation}
	Thus we can introduce another parameter matrix, $ \alpha $ to make a general MC reversible:
	\begin{equation}
	\pi (i) P_{ij} \alpha_{ij} = \pi (j) P_{ji} \alpha_{ji}	,
	\end{equation}
	where $ \alpha_{ij} = \pi(j) P_{ji} $ and $ \alpha_{ji} = \pi(i) P_{ij}$.
	The altered transformation probability matrix is $ P'_{ij} =  P_{ij} \alpha_{ij}$ and $ P'_{ji} =  P_{ji} \alpha_{ji}$, and the MC complies the detailed balance condition: $\pi (i) P'_{ij} = \pi (j) P'_{ji}$.
	The matrix parameter $ \alpha $ is called as ``acceptance rate'', and its physic meaning is as follows: when state $ i $ transforms to state $ j $ with a probability of $ P_{ij} $, the transformation is accepted by the rate of $ \alpha_{ij} $.
	Since the accept rate may be too low for the sampling to move along the MC, we can normalise the $ \alpha $ pair to 1:
	\begin{equation}
		\alpha^{'}(i,j) = min \left\{\frac{\pi(j)P_{ji}}{\pi(i)P_{ij}},1\right\}.
	\end{equation}
	The algorithm is called Metropolis-Hastings and described in following:
	\begin{algorithm}[h]
	  \caption{Metropolis-Hastings Sampling}
	  \label{alg:mcmc}
	  \begin{algorithmic}
	  	
%	    \Procedure{Correction}{coeffs $C$, correlations $Q$}
	    \State Initialisation $x_0 = s_{random}$, \Comment{$ x $:sampling sequence and $s$:state in MC}
	    \For{$t=1, 2, ..., N$}
		    \State $y \sim p_(x \mid x_{t-1})$ 
			    \Comment{Random drawing the next state by the transformation probability matrix $P$}
		    \State $ u \sim Uniform[0,1] $ 
			    \Comment{Random drawing from a uniform distribution}
			\If {$ u < \alpha^{'}(x_{t-1},y) = min \left\{\frac{q(y)p_(x_{t-1} \mid y)}{q(x_{t-1})p_(y \mid x_{t-1})},1\right\} $}
				\State {$x_t = y$} \Comment{Accept the transformation when the random number is less than $\alpha$}
				\Else \State {$x_t = x_{t-1}$}  \Comment{Transformation is refused elsewise}
			\EndIf
		\EndFor
	  \end{algorithmic}
	\end{algorithm}
	
	The probability model function as the product of N normal distributions can be approximated by using this Metropolis-Hastings sampling.
\subsection{Gibbs Sampling}
	For high dimensional data sampling, it is possible to make the accept rate to 1 which increases the convergence speed dramatically.
	According to conditional probability:
	\begin{equation}
		P(A \mid B) = \frac{P(A \cap B)}{P(B)}.
	\end{equation}
	For a $ n $ dimensional data $ (\mathbf{x}, y) $ where $ \mathbf{x}=(x_1,x_2,...,x_{n-1}) $, the joint probability of $p(\mathbf{x},y)$ is:
	\begin{equation}
		p(\mathbf{x},y) = p(y \mid \mathbf{x})p(\mathbf{x}).
	\end{equation}
	Thus, during sampling if we restrict the direction of transformation to one single axis(dimension), $ y $, from point $ A(\mathbf{x}_1, y_1) $ to point $ B(\mathbf{x}_1, y_2)$:
	\begin{equation}
		p(\mathbf{x}_1, y_1)p(y_2 \mid \mathbf{x}_1) = p(\mathbf{x}_1, y_2)p(y_1 \mid \mathbf{x}_1) = p(\mathbf{x}_1)p(y_1 \mid \mathbf{x}_1)p(y_2 \mid \mathbf{x}_1),
	\end{equation}
	then, the MC obeys the condition of detailed balance.
	So the stationary distribution $ \pi(x_1,x_2,...,x_n) $ equals to the joint probability $ p(x_1,x_2,...,x_n) $ and the transformation probability matrix P is consisted of the conditional probability of each dimension $ k $ $,  p(x_k \mid x_1,...,x_{k-1},x_{k+1},...,x_n) $.
	Therefore, given the conditional distribution of each variable for a multivariate distribution Gibbs sampling is able to approximate the joint distribution with long enough sample sequence.
	Gibbs sampling is described as follows:
	\begin{algorithm}[h]
	  \caption{Gibbs Sampling}
	  \label{alg:gibbs}
	  \begin{algorithmic}
	  	
%	    \Procedure{Correction}{coeffs $C$, correlations $Q$}
	    \State Initialisation $\mathbf{x}_0 = [x_0(1),x_0(2),...,x_0(M)]$,  \Comment{Random initialise $\mathbf{x}_0$}
	    \For{$t=1, 2, ..., N$}
	    	\For{$k=1, 2, ..., M$}
	    		\State $ x_t(k) = p(x(k) \mid x_{t-1}(1),x_{t-1}(2),...,x_{t-1}(k-1),x_{t-1}(k+1),...,x_{t-1}(M))$\\
	    		\Comment{Sampling by the conditional distribution}
			\EndFor
		\EndFor
	  \end{algorithmic}
	\end{algorithm}
	
\subsection{CD Instead of KL(Kullback–Leibler)}
	Kullback–Leibler divergence is the measure of how different two probability distributions are:
	\begin{equation}
	\begin{aligned}
%	KL(p^0 \mid \mid p^{\infty})
%	&= \sum_x p^0(\mathbf{x}) \log \dfrac{p^0(\mathbf{x})}{p^{\infty}(\mathbf{x} \mid \Theta)} \\
%	&=  \sum_x p^0(\mathbf{x}) \log p^0(\mathbf{x}) - \left \langle p^{\infty}(\mathbf{x} \mid \Theta) \right \rangle_{p^0} \\
%	&= - H(p^0) + E(\mathbf{x} \mid \Theta),
	KL(P \mid \mid Q)
	&= \sum_x p^0(\mathbf{x}) \log \dfrac{p^0(\mathbf{x})}{p^{\infty}(\mathbf{x} \mid \Theta)} \\
	&=  \sum_x p^0(\mathbf{x}) \log p^0(\mathbf{x}) - \left \langle p^{\infty}(\mathbf{x} \mid \Theta) \right \rangle_{p^0} \\
	&= - H(p^0) + E(\mathbf{x} \mid \Theta),	
	\end{aligned}
	\end{equation}
	and $ - H(p^0)  $ is independent from parameter $ \Theta $ (is static with given data set $ \mathbf{X} $).
	Thus the objective function is the same with Equation~(\ref{equ:like}).
	The partial derivative is the same with Equation~(\ref{equ:part}):
	\begin{equation}
		\label{equ:kl}
		\dfrac{\partial KL(p^0 \mid \mid p^{\infty})}{\partial \theta}
		= \left \langle \dfrac{\partial \log f(\mathbf{x} \mid \Theta)}{\partial \theta}\right \rangle_{\mathbf{p}^\infty} - \left \langle \dfrac{\partial \log f(\mathbf{x} \mid \Theta)}{\partial \theta}\right \rangle_{\mathbf{p}^0}.
	\end{equation}
	Thus,
	\begin{equation}
		\dfrac{\partial [KL(p^0 \mid \mid p^{\infty}) - KL(p^1 \mid \mid p^{\infty})]}{\partial \theta}
		= \left \langle \dfrac{\partial \log f(\mathbf{x} \mid \Theta)}{\partial \theta}\right \rangle_{\mathbf{p}^1} - \left \langle \dfrac{\partial \log f(\mathbf{x} \mid \Theta)}{\partial \theta}\right \rangle_{\mathbf{p}^0},
	\end{equation}
	where the second term in equation~(\ref{equ:kl}) cancels out.
	So Hinton proposed \textcolor{red}{the new contrastive divergence CD to make Gibbs sampling with only 1 step}.
	Contrastive divergence is defined as:
	\begin{equation}
		CD_n = KL(p^0 \mid \mid p^{\infty}) - KL(p^n \mid \mid p^{\infty})
	\end{equation}
\section{RBM\cite{zhang2013rbm}}
	\begin{itemize}
		\item RBM is high dimensional data $ (\mathbf{v}, \mathbf{h}) $, which is perfect for Gibbs sampling.
		$ p(\mathbf{v}, \mathbf{h} \mid \Theta) $ can be approximated by the conditional distribution $ p(\mathbf{v} \mid \mathbf{h}, \Theta) $
		\item RBM is a PoE problem, since
		$ f(\mathbf{v}, \mathbf{h} \mid \Theta) $ is a PoE problem.
		\begin{equation}
			\begin{aligned}
			& f(\mathbf{v}, \mathbf{h} \mid \Theta) =\exp (-E(\mathbf{v}, \mathbf{h} \mid \Theta)) \\
			& E(\mathbf{v}, \mathbf{h} \mid \Theta)= -\sum_{i=1}^n a_i v_i - \sum_{j=1}^m b_j h_j - \sum_{i=1}^n \sum_{j=m}^n v_i W_{ij} h_j.\\
			& p(\mathbf{v}, \mathbf{h} \mid \Theta) =\frac{\exp (E(\mathbf{v}, \mathbf{h} \mid \Theta))}{Z(\Theta)}\\
			& Z(\Theta) = \sum_{\mathbf{v}} \sum_{\mathbf{h}} \exp (-E(\mathbf{v}, \mathbf{h} \mid \Theta)).
			\end{aligned}
		\end{equation}
		Note that the energy function, $ E $, of a RBM is not the same energy function of log of probability product in Equation~(\ref{equ:like}).
		Although, the combined probability function $ p(\mathbf{v}, \mathbf{h} \mid \Theta) $ is nicely defined as a PoE problem, we are more interested in the marginal probability function: $ p(\mathbf{v} \mid \Theta) $.
		$ \mathbf{v} $ is the input of the vision nodes in RBM, which is the training data (not ($ \mathbf{v}, \mathbf{h}$) ). 
		\begin{equation}
		p(\mathbf{v} \mid \Theta) =\frac{\sum_{ \mathbf{h}} \exp (E(\mathbf{v}, \mathbf{h} \mid \Theta))}{Z(\Theta)}
		\end{equation}		
		$ p(\mathbf{v} \mid \Theta) $ is not a PoE problem, but the partial derivation of the negative energy function of the probability product is similar, since the intractable partition function is in common. 
	\end{itemize}
\subsection{Objective Function}
\subsection{CD with 1-step Reconstruction}
\section{Spiking RBM\cite{neftci2013event}}

\begin{appendices}
	\section{derivation process of Equation~(\ref{equ:part})}
	\label{app:part}
		\begin{equation}
		\begin{aligned}
		 \dfrac{\partial \log Z(\Theta)}{\partial \theta}
		=&\frac{1}{Z(\Theta)}\dfrac{\partial Z(\Theta)}{\partial \theta}\\
		=&\frac{1}{Z(\Theta)} \int \dfrac{\partial f(\mathbf{x} \mid \Theta)}{\partial \theta} \D \mathbf{x} \\
		=&\frac{1}{Z(\Theta)} \int f(\mathbf{x} \mid \Theta) \frac{1}{f(\mathbf{x} \mid \Theta)} \dfrac{\partial  f(\mathbf{x} \mid \Theta)}{\partial \theta} \D \mathbf{x}\\
		=& \int \frac{f(\mathbf{x} \mid \Theta) }{Z(\Theta)} \dfrac{\partial \log f(\mathbf{x} \mid \Theta)}{\partial \theta} \D \mathbf{x}\\
		=& \int  p(\mathbf{x} \mid \Theta) \dfrac{\partial \log f(\mathbf{x} \mid \Theta)}{\partial \theta} \D \mathbf{x}\\
		=&\left \langle \dfrac{\partial \log f(\mathbf{c} \mid \Theta)}{\partial \theta}\right \rangle_{C \sim p(\mathbf{x} \mid \Theta)}
		\end{aligned}
		\end{equation}
	\section{  }
\end{appendices}
\bibliography{ref} 
\bibliographystyle{ieeetr}
\end{document}